\section{Sensor systems}
	\subsection{BioHarness}
			The Zephyr Bioharness 3\cite{bioharness} is a physiolocigal monitoring device, which is attached to a strap on the chest. There are several practical uses for this device.
			\begin{itemize}
				\item Remote patient monitoring. Medical patients who need health care but wants to live at home, in combination with ZephyrLIFE \texttrademark \cite{bhpatients}
				\item Sporters who would like to track their progress 
				\item For coaches to find out who is tired and up for substitution during a match. \cite{bhsport}
				\item In 2010 Chilean mineworkers who were trapped underground were remotedly monitored which resulted in a better rescue order and better health care after when the miners work above ground. \cite{chile}
				\item Researchers like me who could use the data.
			\end{itemize}

			The device has an internal storage for more than 500 hours logging and the battery's life cycle is up to 35 hours. The device can be connected to a pc with an USB cable, to transfer data and to recharge the battery. The following logs could be produced: general log, summary log, summary and waveform log and the event log. 
			It takes around 1-6 minutes per hour of data to download the log files from the device to a computer. \cite{bhdatasheet} The data is stored in a folder per session\footnote{A period from which the device is on till it's turned off is called a session.} and within the folder different csv\footnote{csv stands for Comma-Separated Values and such file can be opend with e.g. Microsoft Excel. Zypher also provides scripts to convert csv files to Matlab files.} and DaDisp\footnote{"DADiSP scientific computing and data visualization software that combines the power of programming with the simplicity of a spreadsheet."\cite{dadisp}} files. The sampling frequency of the csv files differs from the sampling frequency of the sensors. Most logs are sampled at \SI{1}{\hertz}.
			The price is 472 dollar.
		
			The accelerometer measures the physical acceleration of the user in the x-, y- and z-axis.. $m/s^2$ is a unit of $g$. The sampling rate is 100 Hz\footnote{Hz stands for hertz and is defined as the number of cycles per second.}. The range is from $-16g$ to $+16g$ for each axis. The acceleration magnitude is $sqrt(X^2+Y^2+Z^2)$.

			The breathing sensor measures the pressure of the chest to the sensor. If the pressure is above a certain threshold it will count as a breath taken. The breath rate is the amount of breaths taken in a minute. The sampling frequency is 25 Hz and the ranges is from 0 to 120.

			The electrocardiography (ECG) sensor measures the electrical activity of the heart.\cite{ECG} See Figure~\ref{fig:ecg} for a graph of the results an ECG procudes. The sampling frequency is 1000 Hz.


			\begin{figure}[h]
				\label{fig:ecg}
				\centering
					\includegraphics[scale=0.5]{ecg.png}
					
				\caption{An graph of 5 seconds of ECG data}

			\end{figure}
			
			% R-R: 250 - 1500 ms, 25 Hz

			% heart rate, R-R interval, breathing rate, ECG, postsure, activity, acceleration

			%The data in stored in a csv file and is an 2D array, for every session a csv file. it's resampled to \SI{1}{Hertz}.
		%Timestamp,HR,BR,Temp,Posture,Activity,Acceleration,Battery,BRAmplitude,ECGAmplitude,ECGNoise,XMin,XPeak,YMin,YPeak,ZMin,ZPeak
	\subsection{Beddit}
		%\subsubsection{General}
		The Beddit Sleep Tracker is placed next to your bed and connected with a sensor between the sheets and the mattress. The makers of Beddit thinks sleep is important, because a human is sleeping one third of his live. Better sleep results in a better life quality. Why not measure it to help improve our sleep quality? The sensor is about \SI{70}{\centi\metre} long and \SI{4}{\centi\metre} wide, so it's possible it will not cover the whole bed. 
			The price is 395 euro.

			The ballistocardiogram (BCG) measures (micro)movements of the body.\cite{beddit}
			BCG is convenient in use, because you won't notice it's measuring, but the disadvantage compared to ECG it's not only measuring cardiac activity, but also body movements, like tossing and turning, which could also be an advantage.\cite{bcg} The sampling frequency is \SI{140}{\hertz}. The devices also measures the room temperature (\SI{}{\celsius}), ambient noise level (\SI{}{\decibel}) and brightness (\SI{}{\lux}), once per 5 minutes.\cite{bedditapi}

			The data is stored in two JSON\footnote{JSON stands for JavaScript Object Notation and is a format to exchange data between different software environments.\cite{json}} files per session.

			\begin{figure}[h]
				\centering
					\includegraphics[scale=0.25]{beddit.jpg}
					
					\caption{The Beddit sensor and the device\cite{beddit}}

			\end{figure}

		%\subsubsection{Description}
		%		Respiration, presence, ihr, actigram, noise, luminosity, temperature, sleep stages, 
		% \subsubsection{Format}
		%	Per day there are two JSON files.
		%	\lstset { 
		%		basicstyle = \footnotesize,
		%		tabsize = 2
		%	}
			%\lstinputlisting[caption=Sleep]{beddit.listing}
			%\lstinputlisting[caption=Results]{beddit2.listing}


	\subsection{Openbeacon}
		%\subsubsection{General}
			RFID stands for Radio-frequency identification and is a technology to communicate wireless using tags. The OpenBeacon Ethernet EasyReader PoE II (called OpenBeacon from now) can receive and send signals with those tags (See figure x for an example). There're two kind of tags, active and passive ones. Active tags are powered by a battery and can broadcast their signal. Passive tags don't have batteries and are powered by the energy received wireless from the OpenBeacon, but only when they're in range. The tags of the Openbeacon are active tags and can communicate with each other. The OpenBeacon Ethernet EasyReader PoE II can identify RFID tags and edges between the tags. Each tag has it's own ID and each edge has a power level. It takes the Openbeacon a few seconds to identify the edges. The price of 100 RFID tags is 1575.75 euro and the EasyReader costs 184.87 euro. RFID tags have a lot of applications, like: anti-theft on clothing, as access key pas to enter buildings, track and trace of goods while transporting, social experiments\cite{2008arXiv0811.4170B}. The OpenBeacon produces JSON output.


			\begin{figure}[h]
				\centering
					\includegraphics[scale=0.5]{tag.jpg}
					\includegraphics[scale=1.0]{reader.jpg}
					
					\caption{A tag for the OpenBeacon\cite{openbeacon} and the OpenBeacon Reader}

			\end{figure}

		%\subsubsection{Description}
		%timestamp, tag1, tag2, power level

		%\subsubsection{Format}
		%The data is stored in a MongoDB in the following format:
		%Tags: ID, Timestamp
		%Edges: Timestamp, Tag1, Tag2
