\section{Conclusions}
	preprocessing a lot of time, 
	need more devices, like calories counter or nike feul.
	More interesting subjects, I'm quite boring, I live to structural.
	Sleep in a structured way improve sleep, 
		but not enough data to really proof correlation between day and night
	Interessted future research, find to partners as subject and see how their lives influence each other.
	This is just a basic experiment and could be really extended by a next student.
	15 days is not enough, more like 15 months. 
	naieve about the data. 
	Still, it is a good setup and could be extended and replaced with maybe easier and better sensors. 
 The models didn't work out well.

	- products dataset
	- lot of data
	-  some insides of sleep
	I have bradycardia. 
	- more features for training set, but not enough rows. ratios.

	at first I had a idea to drink before going to bed coffee one week, and the next week no coffee. This will result in hopefully a good different sleep patterns. But the Data Mining has a paradigm to don't stel op a thesis and collect all the data and see the conclusions afterwards. The problem is, I don't drink coffee, much alcohol, drugs, energy drink or strange things that's could give interessant data because it will interupt/aanpassen my normal life. 

	For another person who has regular shifts maybe better results would came up.


	The beddid is still running every night, and I watch the resuls every day. I improved my life style, not per se because of this project with combining data, but only to watch the statistiscs of my night and watch the advices it's given me. Yesterday I broke my record of deep sleep at the night. I sleep strictly regular, (yes more regular than during the tracking with the bioharness and the openbeacon combined.) and changed some things in my bedroom to make it darker. I also use more noise cancceling earplugs. So it's interessting just by seeing the data makes me want to improve my lifestyle. 

	Quantified jan is a selftracking freak an has a lot of systems, and he putted it all only. Only it are all sepered systems and not combinbed.

	In 3-5 years more and more wearable sensors systems are going to be used by the mainstream. Nike already started the mainstream trend with the Nike Fuel stapperteller. They make it comperable and it would be cool to compare yourself with a sport athelete or just a friend. I see it in front of me, "Hé , you ate last week x calories, but sported just 5 hours, prepare yourself Summer is coming." You will challenge each other to sport. 

	Most of the systems will get an API and websites / communities will come up to combining the data and present it on a gemakkelijk manier. When that time will come it would be interessant to ask for the data and do data mining experiments again. 
