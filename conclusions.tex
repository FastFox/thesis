\section{Conclusions and discussion}
	Working with the three devices is doable for two weeks, but it takes effort to stay focussed and to not forget anything. With more than a million records from multiple data devices, preprocessing takes a lot of time. The code is not advanced, but it is complex because of all the stages the data needs to go through. A lot of loops are used. The way I used the dataset resulted in a trainingset of 15 days and 15 days is not enough to draw trustworthy conclusions, but it is a begin. There are other ways to use the data and if a sampling frequecy of once per every hour is used, it will give 360 hours. The OpenBeacon is nice to have as an extra, but in my situation location can't explain the things I do, because I use the locations for multiple things and most of the time I'm in my bedroom. 
	There are more interesting subjects than me for the experiment. I live structural, don't have health complains, don't smoke, don't do drugs, drink no or little alcohol, drink no coffee and sport a lot. Not a typical student. When a subject is chosen with an unregular live schedule the correlations between the different features could be different, because relations could be more visible. Before the experiment I was thinking about doing something different in one week and doing something else in the other week. It would be interesting to see how I would react. For example drinking coffee before going to bed. However, data mining has a paradigm to don't use hypotheses. If there is a correlation, it will show up eventually.
	\\
	Despite the few days of data, I could agree with the correlations measered. Going later to bed and wake later up will result in more deep sleep, more light sleep and less REM sleep. Based on this I would advice myself to sleep less. With an average of 9 hours and 24 minutes in bed per night, I think that's fair. The interface of Beddit is easy in use and also informative. An advice I read on the website of Beddit was to sleep more structural, even more structural I already did. Now I'm trying to wake up and go to bed every day at the same time. Beddit is still running at the time of writing this and I visit the website of Beddit every day. The statistics motivates me to keep me to my schedule and to improve my life style. I changed the environment to make it darker and sometimes uses earplugs to ignore noise. 
	\\
	In the further more devices could be added or replaced to the experiment. The Nike FuelBand\cite{fuelband} is an activity tracker comparable with the BioHarness. It has an accelerometer, tracks each step taken and also the amount of burned calories. In the next version it will also track the heart rate\cite{fuelband2}. It's more convient to use than the BioHarness, because it's worn on the wrist. Measuring 24/7 will be posible. The OpenBeacon is only be able to get my location in a part of the house. With the help of a smartphone and software it's possible to track someones locations more precise. GPS could be used to get a position outside and Wi-Fi could be used to get a position in an indoor environment\cite{Howard-2003-283}. Fitbit Aria\cite{aria} is a scale and it tracks the weight, body fat percentage and the BMI\footnote{Body Mass Index}. It will also help the user to log the meals and the activity. If theses devices are added to the experiment, with a better subject and for a longer time measurement, more insight will be gained.
	\\
	In 2-4 years more and more wearable sensors systems are going to be used by the mainstream. Nike already started the trend with the Nike FuelBand, but was certainly not the first device in its kind. They have a better marketing system and can reach the normal users. Due to economies of scale, they can produce it for a better price. The product has a small network externality\footnote{A phone has also a network externality. The very first user of a phone had no use of it. He couldn't call anyone else.}. The users of the Nike FualBand like to compare each other and also themselves with top sporters. It also motivated each other to get active. Most of the devices will publish an API\footnote{An Application Programming Interface makes it possible to use the data by external developers.} I predict an social community which will bundle all the API's, will present all the data from the different devices on a user-friendly way and make it shareable. By that time it will interesting to ask the owner of the social community to research the data and do data mining experiments with it. In this project one user is measured for 15 days with 3 devices. Imagine a dataset with 1000+ users with more devices, measered for over a year.
	\\	
	The car insurance company Axa is building a smart phone application which could check the driver's behaviour\cite{axa}. The application will give the drivers advice based on GPS and weather data. The company is researching if it is possible to give the drivers a discount on the premium by good behaviour. I think the next generation of cars could also be connected to a smart phone and more data is available for research. This is just for a driver and his car, but it could also be applied to health insurance. There will be privacy concerns, but eventually customers will take the discount in exchange for their privacy. 
	\\
	 This project is a first approach to use self-tracking with multiple sensor devices. When the demand for these wearable self-tracking sensors is increased the process will get more efficient and more automated. The problem of having not enough data is solved and analyzing the data will be more interesting. 

	\iffalse
	preprocessing a lot of time, 
	need more devices, like calories counter or nike feul.
	More interesting subjects, I'm quite boring, I live to structural.
	Sleep in a structured way improve sleep, 
		but not enough data to really proof correlation between day and night
	Interessted future research, find to partners as subject and see how their lives influence each other.
	This is just a basic experiment and could be really extended by a next student.
	15 days is not enough, more like 15 months. 
	naieve about the data. 
	Still, it is a good setup and could be extended and replaced with maybe easier and better sensors. 
 The models didn't work out well.


	- products dataset
	- lot of data
	-  some insides of sleep
	I have bradycardia. 
	- more features for training set, but not enough rows. ratios.

	at first I had a idea to drink before going to bed coffee one week, and the next week no coffee. This will result in hopefully a good different sleep patterns. But the Data Mining has a paradigm to don't stel op a thesis and collect all the data and see the conclusions afterwards. The problem is, I don't drink coffee, much alcohol, drugs, energy drink or strange things that's could give interessant data because it will interupt/aanpassen my normal life. 
When humans query data we start with an idea, such as: "I think that we sell more DVDs to males than to females." And then we run a query to test the idea and the answer either confirms or disproves our hypothesis. A data mining algorithm doesn't have ideas. It has no intention of testing ideas for the simple reason that it doesn't have any.

	For another person who has regular shifts maybe better results would came up.


	The beddid is still running every night, and I watch the resuls every day. I improved my life style, not per se because of this project with combining data, but only to watch the statistiscs of my night and watch the advices it's given me. Yesterday I broke my record of deep sleep at the night. I sleep strictly regular, (yes more regular than during the tracking with the bioharness and the openbeacon combined.) and changed some things in my bedroom to make it darker. I also use more noise cancceling earplugs. So it's interessting just by seeing the data makes me want to improve my lifestyle. 

	Quantified jan is a selftracking freak an has a lot of systems, and he putted it all only. Only it are all sepered systems and not combinbed.

	In 3-5 years more and more wearable sensors systems are going to be used by the mainstream. Nike already started the mainstream trend with the Nike Fuel stapperteller. They make it comperable and it would be cool to compare yourself with a sport athelete or just a friend. I see it in front of me, "Hé , you ate last week x calories, but sported just 5 hours, prepare yourself Summer is coming." You will challenge each other to sport. 

	Most of the systems will get an API and websites / communities will come up to combining the data and present it on a gemakkelijk manier. When that time will come it would be interessant to ask for the data and do data mining experiments again. 


	There are some correlations, but not strong ones. Not much more days to be certain about the correlation.
	\fi
