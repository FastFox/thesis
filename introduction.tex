\section{Introduction}
	\subsection{Preface}
		This bachelor thesis is part of the track Informatica \& Economie at the Universiteit Leiden. In the first semester I followed the course Data Mining\footnote{A definition of data mining is the process of discovering patterns in data, by analysing the data automatically \cite{datamining}. Another definition is ``the nontrivial extraction of implicit, previously unknown, and potentially useful information from data'' \cite{frawley}.} lectured by Arno Knobbe and got an 8,5 as grade. I became interested in the subject and wanted to do my bachelor project on one of the projects there were offered by the Data Mining Group. I like to sport and do it at least four times a week. I football at svKMD in the first team and cycle at WTOS, a student club. During cycling I am using a Garmin Edge 705 to keep track of my heart rate, speed, average speed and have a spreadsheet with the amount of kilometers per trip. In this project I have combined sport and tracking which resulted in a lot of fun. Thanks Ricardo Cachucho for assisting me and Arno Knobbe for helping and providing the project.

	\subsection{Motivation}
		\label{seq:motivation}
		Actions have consequences and some are clear for us humans and others are not. When something is happening directly after the action and it is consistent, it is easy recognisable as cause-effect. The longer the delay between the cause and effect the less certain we are about the correlation, but we guess the causality is stronger if the pattern happens repeatedly. For example drinking coffee before sleep and laying awake in bed later. Still it is difficult to prove because all other variables that could interrupt sleep should be fixed while experimenting, which is not possible. We could use data mining to help us find causalities about sleep and actions we do during the day. It would give insight how things work and will help improve our life style and quality. In the future maybe it will be possible to explain health complains automatically.

	\subsection{Research Questions}
		Which devices could be used for self-tracking? Which sensors do they have and what do they measure? How does the data look like they produce?
		How to collect the data and bundle the data from the different devices with different sampling frequencies? How can the bundled data be used?
		How to analyse the data? How could self-tracking develop in the future? All these questions will be answered in the paper.
		
		\iffalse
		Which devices could be used for self-tracking? What do they measure?
		How does the data look like? How to bundle the data? How to deal with different sampling frequencies.
		How can the data be used? 
		How can self-tracking develop in the future.
		
		- State of this generation of sensors
		- Different sample frequency
		- Analyze dataset
		- Future
		\fi

		\iffalse
		Self-tracking starts by setting up all the devices. The three devices ?could? have a lot of small practical problems and needs to be tackled. The raw collected data is probably far from useful. The data is formatted in different ways, with different sampling frequencies and different scales. When this is solved it's the question how to analyse the data.

		\

		Self-tracking life, how to setup experiment, difficult to do convenient. A lot of unstructured data. How to structure it so it's useful data. 
		When we have a dataset, how to analyse this data. 

		\

		Life is producing so much data, it's not possible to imagine. But still, every bit more of gathered data could provide interesting insight. We are trying to gather as much as possible data with the available systems and time. Tracking your live must be convenient and should not disturb live, because it will also disturb it's producing data. Setting up such experiment has all kind of practical problems. The raw data is not useable for any data mining algorithm, because the source and format are different. There is much to be done to transform it in a useful dataset. and even this clean dataset can be transformed and restructured to get an extended or derivative dataset. 

		Then there is a discussion which data mining technique can be used. 


		\
			
		Time series, labeled, target

		How to deal with different data systems, collecting the data and working with problems, reorganize the data to get a nice dataset (useable for Data mining), use Data Mining techniques.

		what: self-tracking routines
		How: sensor systems
		Challenge: data integration \& Modeling
		goal: 1 dataset \& model
		\fi
%	\subsection{Motivation}
%		Arno description + personal motivation
%		find correlation between 'pains' and how you lived you day / night. 
%		Try to automaticly analyse the day/night.

	\subsection{Outline}
		In the Sensor Systems section the devices are explained. How do they work, what are the applications, which sensors do they have and what sort of output they produce. In the first part of the Self-Tracking section is explained what the data means, how the data is structered and if it is possible to extract more data. In the second part there is a comparison between two sensors. In the third part is explained how all the data is collected from the systems. In the last part of this section the bundling of the data is explained and how to fix the additional problems. In the Feature Construction section is explained how to extract features from the dataset to get a dataset with a different sampling frequency. In the Data Analysis section the CRSIP-DM model is explained and the data is analysed. Multiple models are made to predict the location of the subject given the physiological data of the day. Also correlations are made, which gives some trivial and some new insights. The last section is about the conclusions, how the self-tracking process could be improved and what the next steps in self-tracking could be.
		
	
	\iffalse
		Klad		


		It's important to explain which systems are being used, what the data means and how the data looks like. Next section there is a comparison between the two physiological devices, what was really useful to learn the devices. The data collection is the most important part of the thesis, because it's explaining how everything was set up and done to get the provided dataset. - Modeling - Conclusions 

		In the Sensor System section the hardware is explained, what the sensors does, what the applications are and how it works. 
		In the experiments section describes useful derivative data, a comparison between the two physiological devices and the experiment to combine all three devices to a good dataset.
		Modeling: I made a dataset of 15 days with all kind of features. Set the deep sleep as target attribute and use WeKa to make a model, with a discussion why deep sleep as target, how the algorithm works and hopefully some insights of the model. 
	\fi
