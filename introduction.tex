\section{Introduction}
	\subsection{Preface}
		This bachelor thesis is part of the track Informatica \& Economie at the Universiteit Leiden. In the first semester I followed the course Data Mining lectured by Arno Knobbe and got an 8,5 as grade. I became interested in the subject and wanted to do my bachelor project on one of the projects there were offered. I like to sport and do this at least four time a week. I football at svKMD in the first team and cycling at WTOS, a student club. During cycling I'm using a Garmin Edge 705 to keep track of my heart rate, speed, average speed and have a spreadsheet with the amount of kilometers per route. In this project I have combined sport and tracking which resulted in a lot of fun. Thanks Ricardo Cachucho for assisting me and Arno Knobbe for providing the project.

	\subsection{Motivation}
		Actions have consequences and some are clear for us humans and others aren't. When something is happening directly after the action and it's consistent, it's easy recognizable as cause-effect. The longer the effect happens after the action the less certain we're about the correlation, but we can still be convinced about our guess about the causality if the it is happens repeatedly. For example drinking coffee before sleep and laying awake in bed later. Still it is difficult to prove because all other variables that could interupt sleep should be fixed while experimenting, which is not possible. We could use data mining to help us find causalities about sleep and actions we do during the day. It would give inside how thinks works and help us improve our life style and quality. Maybe it's in the future possible to explain health complains automatically.

	\subsection{Problem Statement}
		Our live is producing so much data, it's not possible to imagine. But still, every bit more data gathered could provide interesting inside. We are trying to gather as much as possible data, within the trivial boundaries in research. Tracking your live must be convenient and should not disturb live and it's producing data. Setting up such experiment has all kind of practical problems. The raw data is not useable for any data mining algorithm, because the source and format are different. There is much to be done to transform it in a clean dataset, and even this clean dataset can be transformed and restructured to get an extended or derivative dataseta. 

		Then there is a discussion which data mining technique can be used. 


		It's is almost impossible to get a perfect dataset, especially 
			
		Time series, labeled, target

		How to deal with different data systems, collecting the data and working with problems, reorganize the data to get a nice dataset (useable for Data mining), use Data Mining techniques.
	\subsection{Motivation}
		Arno description + personal motivation
		find correlation between 'pains' and how you lived you day / night. 
		Try to automaticly analyze the day/night.

	\subsection{Outline}
		It's important to explain which systems are being used, what the data means and how the data looks like. Next section there is a comparison between the two physiological devices, what was really usefull to learn the devices. The data collection is the most important part of the thesis, because it's explaining how everything was set up and done to get the provided dataset. - Modelling - Conclusions 
		Index in words
